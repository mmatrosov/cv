% Common header for both English and Russian versions

\documentclass[11pt,a4paper,final]{moderncv}

\usepackage[T2A]{fontenc}
\usepackage[utf8]{inputenc}
\usepackage[russian]{babel}
\usepackage{dashundergaps}

\renewcommand{\rmdefault}{cmr}  % Шрифт с засечками
\renewcommand{\sfdefault}{cmss} % Шрифт без засечек
\renewcommand{\ttdefault}{cmtt} % Моноширинный шрифт

% moderncv themes
\moderncvstyle{classic}  % style options are 'casual' (default), 'classic', 'oldstyle' and 'banking'
\moderncvcolor{green}  % color options 'blue' (default), 'orange', 'green', 'red', 'purple', 'grey' and 'black'

\newcommand{\StackExchange}{\includegraphics[height=1em]{stackexchange.png}}

% My custom httplink command with underlined text
\newcommand*{\myhttplink}[2][]{\href{http://#2}{\dotuline{#1}}}

\newcommand{\cvmonth}[1]{{\scriptsize/#1}}

% Language-agnostic personal data
\phone[mobile]{+7~(926)~381-61-64}
\email{mikhail.matrosov@gmail.com}
\social[linkedin]{mmatrosov}
\social[github]{mmatrosov}
\extrainfo{\StackExchange\httplink[Mikhail]{stackoverflow.com/users/261217/mikhail}}


% personal data
\name{Mikhail \vspace{0.5ex} \\}{Matrosov}
\title{C++ developer}                               % optional, remove / comment the line if not wanted
\address{Russia, Moscow} % optional, remove / comment the line if not wanted; the "postcode city" and "country" arguments can be omitted or provided empty
\phone[mobile]{+7~(926)~381-61-64}
\email{mikhail.matrosov@gmail.com}                               % optional, remove / comment the line if not wanted
\social[linkedin]{mmatrosov}                        % optional, remove / comment the line if not wanted
\social[github]{mmatrosov}                              % optional, remove / comment the line if not wanted
\extrainfo{\StackExchange\httplink[Mikhail]{stackoverflow.com/users/261217/mikhail}}

\begin{document}
\maketitle

Interested in C++ applications development and optimization, and in software architecture. Looking for job in an established team of professionals. Image processing area and Windows OS are welcomed.

\section{Skills}

\subsection{Programming languages}

\cvitem{primary}{
	\textbf{C++}: since 2005, image processing applications, multi-threaded applications, console utilities, GUI with Qt and MFC, handling devices with corresponding SDK and via COM-port. \newline
	\textbf{Matlab}: since 2007, algorithms and GUI prototyping, mex-interfaces.
}
\cvitem{secondary}{
	\textbf{C\#}, \textbf{JavaScript}, \textbf{HTML}, \textbf{SQL}: small utilities, simple web-applications.
}

\subsection{Libraries and technologies}

\cvitem{primary}{
	\textbf{Qt}, \textbf{OpenCV}, \textbf{boost}, \textbf{Intel IPP}, \textbf{STL}: multiple image processing projects, problems in areas of combinatorial theory, numerical analysis and graph theory.
}
\cvitem{secondary}{
	\textbf{Microsoft ConcRT}, \textbf{Intel MKL}, \textbf{CGAL}, \textbf{ASP.NET}, \textbf{jQuery}: shallow knowledge, used only in several projects.
}

\subsection{Applications and systems}

\cvitem{primary}{
	\textbf{Windows}, \textbf{Visual Studio}, \textbf{SVN}, \textbf{GIT}, \textbf{win-batch}, \textbf{NuGet}
}
\cvitem{secondary}{
	\textbf{Unix}, \textbf{\LaTeX}, \textbf{Photoshop}, \textbf{bash}
}

\subsection{Scientific and fundamental knowledge}

\cvitem{}{
	Image processing~\cite{matrosov2009diploma}, color theory~\cite{matrosov2013correction}, computer vision, computer graphics, algorithms, data structures.
}


\section{Experience}

\cvitem{2013--2014}{
	\textbf{OctoNus Software Ltd, DM project}, \emph{Developer}. \newline 
	Within DM (Digital Microscope) project, development and support of Qt-based application for visualization and processing of video-stream from network cameras. Elaborated plugins system and SDK. Improved application components interaction.
}
\cvitem{2008--2013}{
	\textbf{OctoNus Software Ltd}, \emph{Developer} \newline 
	Analysis of problems in a jewelry images acquisition software. Image processing algorithms development and integration. Effectively implemented in C++ using Intel IPP and Microsoft ConcRT a number of algorithms of tone mapping, extended depth of field, color correction, image-based 3D-reconstruction and 3D-models stitching. For all the algorithms collected relevant data from partners, analyzed state of the art methods, prototyped solutions in Matlab. Implemented algorithms work in real time and are used in jewelry industry for quality control tasks.
}
\cvitem{2009--2013}{
	\textbf{Graphics and Media Lab, CMC MSU}, \emph{Researcher} \newline{} 
	Member as a PhD student, research activity in OctoNus projects (see above).
}
\cvitem{2011}{
	\textbf{CMC MSU}, \emph{Lecturer} \newline{} 
	C++ laboratory course for students.
}
\cvitem{2004--2006}{
	\textbf{Summer Informatics School}, \emph{Lecturer, Counselor} \newline{} 
	Theoretical and practical courses for group C.
}


\section{Personal projects and contributions}

\cvitem{2014}{
	\myhttplink[\textbf{C++ User Group, Russia}]{meetingcpp.ru}, \emph{Speaker} \newline{} 
	Talk \myhttplink[<<С++ без new и delete>>]{meetingcpp.ru/?p=250}.
}
\cvitem{2012}{
	\myhttplink[\textbf{OpenCV}]{code.opencv.org/users/915}, \emph{Contributor} \newline{} 
	\myhttplink[Patch \#1641]{code.opencv.org/issues/1641}. Discrete Voronoi diagram: returning closest pixel instead of connected component in distanceTransform.
}
\cvitem{2012}{
	\myhttplink[\textbf{NativeViewer}]{sourceforge.net/projects/nativeviewer/}, \emph{Author} \newline{} 
	A Visual Studio extension for visualization of OpenCV images during debug of native C++ applications. Opposed to Microsoft Image Watch, works with Visual Studio versions starting from 2003.
}
\cvitem{since 2010}{
	\myhttplink[\textbf{StackOverflow}]{stackoverflow.com/users/261217/mikhail}, \emph{Contributor} \newline{} 
	Over 6000 reputation, over 50 accepted answers. \newline
	Top 10\% for tags \texttt{c++}, \texttt{algorithm} and \texttt{image-processing}. \newline
	Top 20\% for tags \texttt{matlab} and \texttt{opencv}.
}

\section{Trainings and certificates}

\cvitem{2014}{
	\myhttplink[\textbf{An Overview of the New C++ (C++11/14)}]{www.aristeia.com/C++11.html} \newline{} 
	Intensive technical training by Scott Meyers.
}

\section{Education}

\cvitem{2009--2012}{
	\textbf{Moscow State University} \newline 
	Computational Mathematics and Cybernetics department \newline
	\emph{PhD student}.
}

\cvitem{2004--2009}{
	\textbf{Moscow State University} \newline 
	Computational Mathematics and Cybernetics department \newline
	\emph{Student, specialist}. \newline
	Graduation work~\cite{matrosov2009diploma}. Won admission without matriculation because of I degree diploma in \myhttplink[XVI Russian Olympiad in Informatics]{neerc.ifmo.ru/school/archive/2003-2004/ru-olymp-roi-2004-standings.html}.
}
\cvitem{2001--2003}{
	\textbf{Summer Informatics School} \newline
	\emph{Student of groups С and А}. \newline
	Studying a variety of algorithms and data structures.
}

\section{Languages}

\cvitem{English}{Advanced. Easy reading and writing about an arbitrary topic. Easy conversation about a technical topic.}
\cvitem{Russian}{Native speaker.}


\renewcommand{\bibliographyitemlabel}{[\arabic{enumiv}]}
\renewcommand{\refname}{Publications}
\bibliographystyle{plain}
\bibliography{publications}


\end{document}
