% Common header for both English and Russian versions

\documentclass[11pt,a4paper,final]{moderncv}

\usepackage[T2A]{fontenc}
\usepackage[utf8]{inputenc}
\usepackage[russian]{babel}
\usepackage{dashundergaps}

\renewcommand{\rmdefault}{cmr}  % Шрифт с засечками
\renewcommand{\sfdefault}{cmss} % Шрифт без засечек
\renewcommand{\ttdefault}{cmtt} % Моноширинный шрифт

% moderncv themes
\moderncvstyle{classic}  % style options are 'casual' (default), 'classic', 'oldstyle' and 'banking'
\moderncvcolor{green}  % color options 'blue' (default), 'orange', 'green', 'red', 'purple', 'grey' and 'black'

\newcommand{\StackExchange}{\includegraphics[height=1em]{stackexchange.png}}

% My custom httplink command with underlined text
\newcommand*{\myhttplink}[2][]{\href{http://#2}{\dotuline{#1}}}

\newcommand{\cvmonth}[1]{{\scriptsize/#1}}

% Language-agnostic personal data
\phone[mobile]{+7~(926)~381-61-64}
\email{mikhail.matrosov@gmail.com}
\social[linkedin]{mmatrosov}
\social[github]{mmatrosov}
\extrainfo{\StackExchange\httplink[Mikhail]{stackoverflow.com/users/261217/mikhail}}


\usepackage{tipa}

% Language-specific personal data
\name{Mikhail\textsuperscript{*} \vspace{0.5ex} \\}{Matrosov}
\title{C++ developer}
\address{Russia, Moscow}


\begin{document}
\maketitle

C++ jedi. Create software and know what teamwork is. Solve problems and optimize everything. Good at communicating with people and love to teach people. About me, in three words: C++, GIT, Visual Studio.
\par\medskip
* Letter ``k'' is not pronounced in my name, it is \textipa{[{m\super j}Ix5{\textprimstress}il]}, like \myhttplink[Gorbachev]{http://en.wikipedia.org/wiki/Mikhail_Gorbachev}.

\section{Skills}

\subsection{Programming languages}

\cvitem{primary}{
	\textbf{C++}: from 2005, image processing applications, multi-threaded applications, console utilities, GUI with Qt and MFC, 3D-graphics, CAD systems; making internal and public technical talks and seminars. \newline
	\textbf{Matlab}: from 2007 till 2014, algorithms and GUI prototyping, mex-interfaces.
}
\cvitem{secondary}{
	\textbf{Python}, \textbf{C\#}, \textbf{JavaScript}, \textbf{SQL}: small utilities, home projects, simple web-applications.
}

\subsection{Libraries and technologies}

\cvitem{primary}{
	\textbf{STL}, \textbf{boost}, \textbf{OpenCV}, \textbf{Qt}, \textbf{Intel IPP}: multiple image processing projects, problems in areas of combinatorial theory, numerical analysis, graph theory, 2D and 3D geometry.
}
\cvitem{secondary}{
	\textbf{Microsoft ConcRT}, \textbf{Intel MKL}, \textbf{OpenSceneGraph}, \textbf{CGAL}, \textbf{ASP.NET}, \textbf{jQuery}: shallow knowledge, used only in several projects.
}

\subsection{Applications and systems}

\cvitem{primary}{
	\textbf{Windows}, \textbf{Visual Studio}, \textbf{SmartGit}, \textbf{GIT}, \textbf{SVN}, \textbf{Total Commander}, \textbf{Conan}
}
\cvitem{secondary}{
	\textbf{Unix}, \textbf{\LaTeX}, \textbf{Photoshop}, \textbf{bash}
}

\subsection{Scientific and fundamental knowledge}

\cvitem{}{
	Image processing~\cite{matrosov2009diploma}, color theory~\cite{matrosov2013correction}, computer vision, computer graphics, algorithms, data structures.
}

\subsection{Productivity}

\cvitem{}{
	Touch typing in both English and Russian layouts. To optimize movements of the right hand during typing, created a special utility called KeysRemapper (see ``\hyperref[sec:contributions]{Contributions}'' section).
}


\section{Experience}

\cvitem{from~2018\cvmonth{02}}{
	\textbf{Align Technology Inc}, \emph{Software Development Expert}. \newline 
	Continue work in 3D platform team. Improve management of C++ third-party libraries: switching from a custom solution to Conan package manager. Technical guidance and mentoring of team of junior developers working on cross-platform support for a subset of applications. Upgrade compiler from Visual Studio 2013 to Visual Studio 2017.
}
\cvitem{from~2017\cvmonth{01} till~2018\cvmonth{02}}{
	\textbf{Align Technology Inc}, \emph{Technical Manager, 3D platform team}. \newline 
	Switch completely to global software problems, affecting all developers. Migration to 64-bit platform. Support for localization. Support for Unicode. Unification of code formatting. Architectural rework: model/view separation. Vision of the future software evolution. Plus enhancement of development process as before.
}
\cvitem{from~2016\cvmonth{02} till~2016\cvmonth{12}}{
	\textbf{Align Technology Inc}, \emph{Technical Manager}. \newline 
	Business analysis: gathering of requirements for new features from all positions including software end users and doctors. Cross-command communication for successful integration of features in all components. Company promotion: public talks and blog posts. Plus all the same as before.
}
\cvitem{from~2014\cvmonth{12} till~2016\cvmonth{02}}{
	\textbf{Align Technology Inc}, \emph{Senior developer}. \newline 
	Developing CAD-application for orthodontic treatment planning. Managing development of features to enter teen segment of the market. Enhancing development process: code review, GIT workflow, coding standards, static analysis. Geometric algorithms on solid 3D bodies. Code base modernization.
}
\cvitem{from~2013\cvmonth{10} till~2014\cvmonth{10}}{
	\textbf{OctoNus Software Ltd, Digital Microscope project}, \newline
	\emph{Developer}. \newline
	Prototyped virtual navigation system through a number of photographs of an object, system has six degrees of freedom. Developed color correction algorithm, based on a color space continuous transform through a number of pivots. Did development and support of Qt-based application for visualization and processing of video-stream from network cameras. Elaborated plugins system and SDK. Improved application components interaction.
}
\cvitem{from~2008\cvmonth{08} till~2013\cvmonth{09}}{
	\textbf{OctoNus Software Ltd}, \emph{Developer} \newline 
	Analysis of problems in a jewelry images acquisition software. Image processing algorithms development and integration. Effectively implemented in C++ using Intel IPP and Microsoft ConcRT a number of algorithms of tone mapping, extended depth of field, color correction, image-based 3D-reconstruction and 3D-models stitching. Developed lens breath compensation algorithm. For all the algorithms collected relevant data from partners, analyzed state of the art methods, prototyped solutions in Matlab. Implemented algorithms work in real time and are used in jewelry industry for quality control tasks.
}
\cvitem{from~2009\cvmonth{10} till~2013\cvmonth{10}}{
	\textbf{Graphics and Media Lab, CMC MSU}, \emph{Researcher} \newline 
	Member as a PhD student, research activity in OctoNus projects (see above).
}
\cvitem{from~2011\cvmonth{02} till~2011\cvmonth{05}}{
	\textbf{CMC MSU}, \emph{Lecturer} \newline 
	C++ laboratory course for students.
}
\cvitem{from~2004 till~2006}{
	\textbf{Summer Informatics School}, \emph{Lecturer, Counselor} \newline 
	Theoretical and practical courses for group C.
}


\section{Public talks}

Talks marked with $\star$ symbol deserve to be highlighted. Talks with captions in English are in English.

\cvitem{$\star$ 2017\cvmonth{10}}{
	\myhttplink[\textbf{SECR 2018}]{https://2018.secrus.org/lang/en/} \newline 
	\myhttplink[``Как не потонуть в пучине легаси'']{https://2018.secrus.org/program/submitted-presentations/how-not-to-sink-in-legacy/}
}
\cvitem{2018\cvmonth{02}}{
	\myhttplink[\textbf{C++ Russia 2018}]{http://2018.cppconf.ru} \newline 
	\myhttplink[``Versatile C++ applied'']{https://2018.cppconf.ru/talks/mikhail-matrosov.html}
}
\cvitem{$\star$ 2017\cvmonth{10}}{
	\myhttplink[\textbf{SECR 2017}]{http://2017.secr.ru/lang/en/} \newline 
	\myhttplink[``Reverting a merge'']{http://2017.secr.ru/lang/en/program/submitted-presentations/reverting-a-merge-without-console}
}
\cvitem{$\star$ 2017\cvmonth{09}}{
	\myhttplink[\textbf{CppCon 2017}]{https://cppcon.org/cppcon-2017-program/} \newline 
	\myhttplink[``Refactor or die'']{https://youtu.be/fzmjXK9JZ9o}
}
\cvitem{2017\cvmonth{04}}{
	\myhttplink[\textbf{SECON 2017}]{http://2017.secon.ru} \newline 
	\myhttplink[``Повседневный С++: алгоритмы и итераторы'']{http://2017.secon.ru/reports/povsednevnyy-s-algoritmy-i-iteratory}
}
\cvitem{$\star$ 2017\cvmonth{02}}{
	\myhttplink[\textbf{C++ Russia 2017}]{http://2017.cppconf.ru} \newline 
	\myhttplink[``Повседневный С++: алгоритмы и итераторы'']{http://2017.cppconf.ru/talks/mikhail-matrosov}
}
\cvitem{2017\cvmonth{02}}{
	\myhttplink[\textbf{C++ CoreHard Winter 2017 Сonference}]{http://corehard.by/category/corehard-conf-winter-2017/} \newline 
	\myhttplink[``Повседневный С++: алгоритмы и итераторы'']{http://corehard.by/2017/02/20/day-to-day-c-algorithms-and-iterators/}
}
\cvitem{$\star$ 2016\cvmonth{02}}{
	\myhttplink[\textbf{C++ Russia 2016 Conference}]{http://cpp-russia.ru/?page_id=936} \newline 
	\myhttplink[``Повседневный С++: boost и STL'']{http://cpp-russia.ru/?page_id=999}
}
\cvitem{2016\cvmonth{02}}{
	\myhttplink[\textbf{C++ Corehard Conf 2016}]{http://corehard.by/category/corehard-conf-2016/} \newline 
	\myhttplink[``Повседневный С++: boost и STL'']{http://corehard.by/2016/02/15/conf2016-daily-cpp/}
}
\cvitem{2015\cvmonth{10}}{
	\myhttplink[\textbf{Conference CEE-SECR 2015}]{http://2015.secr.ru/lang/en} \newline 
	\myhttplink[``Повседневный С++'']{http://2015.secr.ru/lang/en/program/submitted-presentations/daily-cpp}
}
\cvitem{$\star$ 2015\cvmonth{02}}{
	\myhttplink[\textbf{C++ Russia 2015 Conference}]{http://cpp-russia.ru/?page_id=233} \newline 
	\myhttplink[``С++ without new and delete'']{http://cpp-russia.ru/?page_id=608}
}
\cvitem{2014\cvmonth{10}}{
	\myhttplink[\textbf{Meeting of C++ User Group, Russia}]{http://cpp-russia.ru/?p=286} \newline 
	\myhttplink[``С++ без new и delete'']{http://www.slideshare.net/sermp/c-without-new-and-delete-russian-c-user-group}
}


\section{Personal projects and contributions}\label{sec:contributions}

\cvitem{from~2019\cvmonth{02} till~2019\cvmonth{5}}{
	\myhttplink[\textbf{Основы разработки на С++: чёрный пояс}]{https://www.coursera.org/learn/c-plus-plus-black}, \newline \emph{Coauthor} \newline 
	A coursera course created together with Yandex. Fifth out of five in the specialization \myhttplink[Искусство разработки на современном C++]{https://www.coursera.org/specializations/c-plus-plus-modern-development}. In my block I teach about undefined behavior and go through a big trainig problem of creating a spreadsheets engine (like Excel and backend for Google Spreadsheets).
}
\cvitem{from~2018\cvmonth{09} till~2018\cvmonth{12}}{
	\myhttplink[\textbf{Основы разработки на С++: коричневый пояс}]{https://www.coursera.org/learn/c-plus-plus-brown}, \newline \emph{Coauthor} \newline 
	A coursera course created together with Yandex. Fourth out of five in the specialization \myhttplink[Искусство разработки на современном C++]{https://www.coursera.org/specializations/c-plus-plus-modern-development}. Has 5.0 rating in three month after the launch. In my block I teach about Smart Pointers.
}
\cvitem{2017\cvmonth{05}}{
	\myhttplink[\textbf{DllDispatcher}]{https://github.com/mmatrosov/DllDispatcher}, \emph{Author} \newline 
	The tool that allows to associate dll files with different applications based on dll bitness. One application for 32-bit dlls and another for 64-bit dlls. Designed for associating dlls with corresponding versions of Dependency Walker.
}
\cvitem{2015\cvmonth{03}}{
	\myhttplink[\textbf{boost.python}]{http://boost.org/libs/python}, \emph{Contributor} \newline 
	\myhttplink[Pull request \#15]{http://github.com/boostorg/python/pull/15}. Fix \#11100 and \#8058: binary compatibility and leaked file handle in exec\_file().
}
\cvitem{2014\cvmonth{10}}{
	\myhttplink[\textbf{KeysRemapper}]{http://github.com/mmatrosov/KeysRemapper}, \emph{Author} \newline 
	When CapsLock is switched on, this utility turns a part of alphabetic keyboard into navigation buttons: arrows, Ins, Del, Home, End, etc. Such a simplified Vim, but working in the entire OS, not in text editor only.
}
\cvitem{from~2012\cvmonth{06} till~2013\cvmonth{07}}{
	\myhttplink[\textbf{NativeViewer}]{http://sourceforge.net/projects/nativeviewer/}, \emph{Author} \newline 
	A Visual Studio extension for visualization of OpenCV images during debug of native C++ applications. Opposed to Microsoft Image Watch, works with Visual Studio versions starting from 2003.
}
\cvitem{2012\cvmonth{02}}{
	\myhttplink[\textbf{OpenCV}]{http://code.opencv.org/users/915}, \emph{Contributor} \newline 
	\myhttplink[Patch \#1641]{http://code.opencv.org/issues/1641}. Discrete Voronoi diagram: returning closest pixel instead of connected component in distanceTransform.
}
\cvitem{from~2010\cvmonth{01}}{
	\myhttplink[\textbf{StackOverflow}]{http://stackoverflow.com/users/261217/mikhail}, \emph{Contributor} \newline 
	Over 10k reputation, \myhttplink[over 80 accepted answers]{http://stackoverflow.com/search?q=user:261217+isaccepted:yes}. \newline
	Top 10\% for tags \texttt{c++}, \texttt{algorithm} and \texttt{image-processing}. \newline
	Top 20\% for tags \texttt{matlab} and \texttt{opencv}.
}

\section{Trainings and certificates}

\cvitem{2019\cvmonth{04}}{
	\myhttplink[\textbf{The Bits and Bytes of Computer Networking}]{https://www.coursera.org/learn/computer-networking} \newline 
	Networking basics from Google. \myhttplink[Link to the certificate]{https://www.coursera.org/account/accomplishments/verify/D6G7BH6YMS8N}.
}
\cvitem{2014\cvmonth{06}}{
	\myhttplink[\textbf{An Overview of the New C++ (C++11/14)}]{http://www.aristeia.com/C++11.html} \newline 
	Intensive technical training by Scott Meyers.
}

\section{Education}

\cvitem{from~2009 till~2012}{
	\textbf{Moscow State University} \newline 
	Computational Mathematics and Cybernetics department \newline
	\emph{PhD student}.
}
\cvitem{from~2004 till~2009}{
	\textbf{Moscow State University} \newline 
	Computational Mathematics and Cybernetics department \newline
	\emph{Student, specialist}. \newline
	Graduation work~\cite{matrosov2009diploma}. Won admission without matriculation because of I degree diploma in \myhttplink[XVI Russian Olympiad in Informatics]{http://neerc.ifmo.ru/school/archive/2003-2004/ru-olymp-roi-2004-standings.html}.
}
\cvitem{from~2001 till~2003}{
	\textbf{Summer Informatics School} \newline
	\emph{Student of groups С and А}. \newline
	Studying a variety of algorithms and data structures.
}

\section{Languages}

\cvitem{English}{Advanced. Easy reading and writing about an arbitrary topic. Easy conversation about a technical topic.}
\cvitem{Russian}{Native speaker.}


\renewcommand{\bibliographyitemlabel}{[\arabic{enumiv}]}
\renewcommand{\refname}{Publications}
\bibliographystyle{plain}
\bibliography{publications}


\end{document}
