% Common header for both English and Russian versions

\documentclass[11pt,a4paper,final]{moderncv}

\usepackage[T2A]{fontenc}
\usepackage[utf8]{inputenc}
\usepackage[russian]{babel}
\usepackage{dashundergaps}

\renewcommand{\rmdefault}{cmr}  % Шрифт с засечками
\renewcommand{\sfdefault}{cmss} % Шрифт без засечек
\renewcommand{\ttdefault}{cmtt} % Моноширинный шрифт

% moderncv themes
\moderncvstyle{classic}  % style options are 'casual' (default), 'classic', 'oldstyle' and 'banking'
\moderncvcolor{green}  % color options 'blue' (default), 'orange', 'green', 'red', 'purple', 'grey' and 'black'

\newcommand{\StackExchange}{\includegraphics[height=1em]{stackexchange.png}}

% My custom httplink command with underlined text
\newcommand*{\myhttplink}[2][]{\href{http://#2}{\dotuline{#1}}}

\newcommand{\cvmonth}[1]{{\scriptsize/#1}}

% Language-agnostic personal data
\phone[mobile]{+7~(926)~381-61-64}
\email{mikhail.matrosov@gmail.com}
\social[linkedin]{mmatrosov}
\social[github]{mmatrosov}
\extrainfo{\StackExchange\httplink[Mikhail]{stackoverflow.com/users/261217/mikhail}}


% personal data
\name{Mikhail \vspace{0.5ex} \\}{Matrosov}
\title{C++ developer}                               % optional, remove / comment the line if not wanted
\address{Russia, Moscow} % optional, remove / comment the line if not wanted; the "postcode city" and "country" arguments can be omitted or provided empty
\phone[mobile]{+7~(926)~381-61-64}
\email{mikhail.matrosov@gmail.com}                               % optional, remove / comment the line if not wanted
\social[linkedin]{mmatrosov}                        % optional, remove / comment the line if not wanted
\social[github]{mmatrosov}                              % optional, remove / comment the line if not wanted
\extrainfo{\StackExchange\httplink[Mikhail]{stackoverflow.com/users/261217/mikhail}}

\begin{document}
\maketitle

Interested in C++ applications development and optimization, and in software architecture. Looking for job in an established team of professionals. Image processing area and Windows OS are welcomed.

\section{Skills}

\subsection{Programming languages}

\cvitem{primary}{
	\textbf{C++}: from 2005, image processing applications, multi-threaded applications, console utilities, GUI with Qt and MFC, 3D-graphics; making internal and public technical talks and seminars. \newline
	\textbf{Matlab}: from 2007 till 2014, algorithms and GUI prototyping, mex-interfaces.
}
\cvitem{secondary}{
	\textbf{Python}, \textbf{C\#}, \textbf{JavaScript}, \textbf{HTML}, \textbf{SQL}: small utilities, home projects, simple web-applications.
}

\subsection{Libraries and technologies}

\cvitem{primary}{
	\textbf{STL}, \textbf{boost}, \textbf{Qt}, \textbf{OpenCV}, \textbf{Intel IPP}: multiple image processing projects, problems in areas of combinatorial theory, numerical analysis, graph theory, 2D and 3D geometry.
}
\cvitem{secondary}{
	\textbf{Microsoft ConcRT}, \textbf{Intel MKL}, \textbf{CGAL}, \textbf{ASP.NET}, \textbf{jQuery}: shallow knowledge, used only in several projects.
}

\subsection{Applications and systems}

\cvitem{primary}{
	\textbf{Windows}, \textbf{Visual Studio}, \textbf{SmartGit}, \textbf{GIT}, \textbf{SVN}, \textbf{win-batch}, \textbf{NuGet}
}
\cvitem{secondary}{
	\textbf{Unix}, \textbf{\LaTeX}, \textbf{Photoshop}, \textbf{bash}
}

\subsection{Scientific and fundamental knowledge}

\cvitem{}{
	Image processing~\cite{matrosov2009diploma}, color theory~\cite{matrosov2013correction}, computer vision, computer graphics, algorithms, data structures.
}


\section{Experience}

\cvitem{from~2014/12}{
	\textbf{Align Technology Inc}, \emph{Senior developer}. \newline 
	Developing 3D-application for designing personalized orthodontic appliances (aligners). Managing of technical perspectives of new products aimed for marked expansion. Enhancing development process: code review, GIT workflow, coding standards. 
}
\cvitem{from~2013/10 till~2014/10}{
	\textbf{OctoNus Software Ltd, Digital Microscope project}, \newline
	\emph{Developer}. \newline
	Prototyped virtual navigation system though a number of photographs of an object, system has six degrees of freedom. Developed color correction algorithm, based on a color space continuous transform through a number of pivots. Did development and support of Qt-based application for visualization and processing of video-stream from network cameras. Elaborated plugins system and SDK. Improved application components interaction.
}
\cvitem{from~2008/08 till~2013/09}{
	\textbf{OctoNus Software Ltd}, \emph{Developer} \newline 
	Analysis of problems in a jewelry images acquisition software. Image processing algorithms development and integration. Effectively implemented in C++ using Intel IPP and Microsoft ConcRT a number of algorithms of tone mapping, extended depth of field, color correction, image-based 3D-reconstruction and 3D-models stitching. For all the algorithms collected relevant data from partners, analyzed state of the art methods, prototyped solutions in Matlab. Implemented algorithms work in real time and are used in jewelry industry for quality control tasks.
}
\cvitem{from~2009/10 till~2013/10}{
	\textbf{Graphics and Media Lab, CMC MSU}, \emph{Researcher} \newline 
	Member as a PhD student, research activity in OctoNus projects (see above).
}
\cvitem{from~2011/02 till~2011/05}{
	\textbf{CMC MSU}, \emph{Lecturer} \newline 
	C++ laboratory course for students.
}
\cvitem{from~2004 till~2006}{
	\textbf{Summer Informatics School}, \emph{Lecturer, Counselor} \newline 
	Theoretical and practical courses for group C.
}


\section{Personal projects and contributions}

\cvitem{2015/02}{
	\myhttplink[\textbf{C++ Russia Conference}]{meetingcpp.ru/?page_id=233}, \emph{Speaker} \newline 
	Talk \myhttplink[<<С++ without new and delete>>]{meetingcpp.ru/?page_id=608}.
}
\cvitem{2014/10}{
	\myhttplink[\textbf{Meeting of C++ User Group, Russia}]{meetingcpp.ru/?p=250}, \emph{Speaker} \newline 
	Talk \myhttplink[<<С++ без new и delete>>]{www.slideshare.net/sermp/c-without-new-and-delete-russian-c-user-group}.
}
\cvitem{2012/02}{
	\myhttplink[\textbf{OpenCV}]{code.opencv.org/users/915}, \emph{Contributor} \newline 
	\myhttplink[Patch \#1641]{code.opencv.org/issues/1641}. Discrete Voronoi diagram: returning closest pixel instead of connected component in distanceTransform.
}
\cvitem{from~2012/06 till~2013/07}{
	\myhttplink[\textbf{NativeViewer}]{sourceforge.net/projects/nativeviewer/}, \emph{Author} \newline 
	A Visual Studio extension for visualization of OpenCV images during debug of native C++ applications. Opposed to Microsoft Image Watch, works with Visual Studio versions starting from 2003.
}
\cvitem{from~2010/01}{
	\myhttplink[\textbf{StackOverflow}]{stackoverflow.com/users/261217/mikhail}, \emph{Contributor} \newline 
	Over 7500 reputation, over 50 accepted answers. \newline
	Top 10\% for tags \texttt{c++}, \texttt{algorithm} and \texttt{image-processing}. \newline
	Top 20\% for tags \texttt{matlab} and \texttt{opencv}.
}

\section{Trainings and certificates}

\cvitem{2014/06}{
	\myhttplink[\textbf{An Overview of the New C++ (C++11/14)}]{www.aristeia.com/C++11.html} \newline 
	Intensive technical training by Scott Meyers.
}

\section{Education}

\cvitem{from~2009 till~2012}{
	\textbf{Moscow State University} \newline 
	Computational Mathematics and Cybernetics department \newline
	\emph{PhD student}.
}
\cvitem{from~2004 till~2009}{
	\textbf{Moscow State University} \newline 
	Computational Mathematics and Cybernetics department \newline
	\emph{Student, specialist}. \newline
	Graduation work~\cite{matrosov2009diploma}. Won admission without matriculation because of I degree diploma in \myhttplink[XVI Russian Olympiad in Informatics]{neerc.ifmo.ru/school/archive/2003-2004/ru-olymp-roi-2004-standings.html}.
}
\cvitem{from~2001 till~2003}{
	\textbf{Summer Informatics School} \newline
	\emph{Student of groups С and А}. \newline
	Studying a variety of algorithms and data structures.
}

\section{Languages}

\cvitem{English}{Advanced. Easy reading and writing about an arbitrary topic. Easy conversation about a technical topic.}
\cvitem{Russian}{Native speaker.}


\renewcommand{\bibliographyitemlabel}{[\arabic{enumiv}]}
\renewcommand{\refname}{Publications}
\bibliographystyle{plain}
\bibliography{publications}


\end{document}
