\documentclass[11pt,a4paper,final]{moderncv}

\usepackage[T2A]{fontenc}
% исходный текст в кодировки unicode
\usepackage[utf8]{inputenc}
% включаем поддержку русского языка
\usepackage[russian]{babel}

\renewcommand{\rmdefault}{cmr} % Шрифт с засечками
\renewcommand{\sfdefault}{cmss} % Шрифт без засечек
\renewcommand{\ttdefault}{cmtt} % Моноширинный шрифт

% moderncv themes
\moderncvstyle{classic}                             % style options are 'casual' (default), 'classic', 'oldstyle' and 'banking'
\moderncvcolor{green}                               % color options 'blue' (default), 'orange', 'green', 'red', 'purple', 'grey' and 'black'

\newcommand{\StackExchange}{\includegraphics[height=1em]{stackexchange.png}}

% personal data
\name{Михаил \vspace{0.5ex} \\}{Матросов}
\title{C++ разработчик}                               % optional, remove / comment the line if not wanted
\address{Россия, Москва} % optional, remove / comment the line if not wanted; the "postcode city" and "country" arguments can be omitted or provided empty
\phone[mobile]{+7~(926)~381-61-64}
\email{mikhail.matrosov@gmail.com}                               % optional, remove / comment the line if not wanted
\social[linkedin]{mmatrosov}                        % optional, remove / comment the line if not wanted
\social[github]{mmatrosov}                              % optional, remove / comment the line if not wanted
\extrainfo{\StackExchange \httplink[Mikhail]{stackoverflow.com/users/261217/mikhail}}

\begin{document}
\maketitle

\section{Коротко о главном}

Большой опыт разработки приложений по обработке изображений на С++ в среде Visual Studio. Хорошее владение Matlab для прототипирования проектов. Много работал с библиотеками Intel IPP и OpenCV. Базовые представления об оптимизации многопоточных программ.

Большой опыт исследовательской работы в области обработки изображений, тональной компрессии и теории цвета. Тем не менее, больше заинтересован в развитии технических навыков, чем в продолжении научной работы.

Богатое прошлое в олимпиадном программировании. Поступил в МГУ без экзаменов благодаря диплому I степени на \httplink[XVI Всероссийской Олимпиаде по Информатике]{neerc.ifmo.ru/school/archive/2003-2004/ru-olymp-roi-2004-standings.html}. Хорошее знание широкого класса алгоритмов и структур данных.

\section{Навыки}

\subsection{Языки программирования}

\cvitem{основные}{
	\textbf{C++}: 7 лет разработки, приложения по обработке изображений, многопоточные приложения, консольные утилиты, GUI на MFC, работа с устройствами через соответствующие SDK и COM-порт. \newline
	\textbf{Matlab}: 5 лет разработки, прототипирование приложений, mex-интерфейсы.
}
\cvitem{работал с}{
	\textbf{C\#}, \textbf{JavaScript}, \textbf{HTML}, \textbf{SQL}, \textbf{PHP}: поверхностные знания для решения прикладных задач.
}

\subsection{Библиотеки и технологии}

\cvitem{основные}{
	\textbf{OpenCV}, \textbf{Intel IPP}: множество проектов по обработке изображений.\newline
	\textbf{STL}, \textbf{Microsoft ConcRT}: проекты с использованием комбинаторики, численных методов, теории графов.
}
\cvitem{работал с}{
	\textbf{boost}, \textbf{Intel MKL}, \textbf{CGAL}, \textbf{ASP.NET}, \textbf{jQuery}: поверхностные знания для решения прикладных задач.
}

\subsection{Прикладные программы и системы}

\cvitem{основные}{
	\textbf{Visual Studio}, \textbf{SVN}, \textbf{GIT}, \textbf{win-batch}
}
\cvitem{работал с}{
	\textbf{\LaTeX}, \textbf{Photoshop}, \textbf{bash}
}

\subsection{Научные и фундаментальные знания}

\cvitem{}{
	Обработка изображений~\cite{matrosov2009diploma}, теория цвета~\cite{matrosov2013correction}, компьютерное зрение, машинная графика, алгоритмы, структуры данных.
}


\section{Опыт работы}

\cvitem{c 2009}{
	\textbf{OctoNus Software Ltd.} \newline{} 
	Разработка алгоритмов и ПО. Основной опыт работы. Анализ задач клиентов, связанных с качеством изображений ювелирной продукции, разработка методик и алгоритмов для их решения. Реализация и оптимизация алгоритмов, внедрение методик в производственный процесс.
}
\cvitem{c 2009}{
	\textbf{Лаборатория Компьютерной Графики и Мультимедиа ВМК МГУ} \newline{} 
	Исследовательская работа в качестве аспиранта. Тесно связана с работой в OctoNus.
}
\cvitem{2011}{
	\textbf{Кафедра АСВК факультета ВМК МГУ} \newline{} 
	Преподаватель практикума по С++ для студентов 3 курса.
}
\cvitem{2004--2006}{
	\textbf{Летняя Компьютерная Школа} \newline{} 
	Преподаватель группы С, вожатый. 
}


\section{Свои проекты и вклады}

\cvitem{с 2012}{
	\httplink[\textbf{NativeViewer}]{sourceforge.net/projects/nativeviewer/}, \emph{автор} \newline{} 
	Расширение Visual Studio для просмотра изображений OpenCV прямо во время отладки С++ кода. В отличие от Microsoft Image Watch, работает для всех версий Visual Studio.
}
\cvitem{с 2010}{
	\httplink[\textbf{StackOverflow}]{http://stackoverflow.com/users/261217/mikhail}, \emph{участник} \newline{} 
	Более 3000 репутации, более 30 принятых ответов. \newline{} 
	Топ 10\% по тэгам algorithm и image-processing. \newline{} 
	Топ 20\% по тэгам c++, matlab и opencv.
}
\cvitem{2012}{
	\httplink[\textbf{OpenCV}]{http://code.opencv.org/users/915}, \emph{участник} \newline{} 
	\httplink[Патч \#1641]{code.opencv.org/issues/1641}. Discrete Voronoi diagram: returning closest pixel instead of connected component in distanceTransform.
}

\section{Образование}

\cvitem{2009--2012}{
	\textbf{Московский Государственный Университет} \newline{} 
	Факультет Вычислительной Математики и Кибернетики \newline{} 
	Аспирант кафедры АСВК.
}

\cvitem{2004--2009}{
	\textbf{Московский Государственный Университет} \newline{} 
	Факультет Вычислительной Математики и Кибернетики \newline{} 
	Студент, специалист. Диплом~\cite{matrosov2009diploma}.
}
\cvitem{2002--2003}{
	\textbf{Летняя Компьютерная Школа} \newline{} 
	Ученик групп С и А. Изучение широкого класса алгоритмов и структур данных.
}

\section{Владение языками}

\cvitem{Английский}{Продвинутый. Свободное чтение и письмо на произвольные темы. Свободный диалог на технические темы.}
\cvitem{Русский}{Носитель. Грамотное письмо.}



\renewcommand{\bibliographyitemlabel}{[\arabic{enumiv}]}
\renewcommand{\refname}{Публикации}
\bibliographystyle{plain}
\bibliography{publications}


\end{document}
