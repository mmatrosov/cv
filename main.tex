\documentclass[11pt,a4paper,final]{moderncv}

\usepackage[T2A]{fontenc}
% исходный текст в кодировки unicode
\usepackage[utf8]{inputenc}
% включаем поддержку русского языка
\usepackage[russian]{babel}

\renewcommand{\rmdefault}{cmr} % Шрифт с засечками
\renewcommand{\sfdefault}{cmss} % Шрифт без засечек
\renewcommand{\ttdefault}{cmtt} % Моноширинный шрифт

% moderncv themes
\moderncvstyle{casual}                             % style options are 'casual' (default), 'classic', 'oldstyle' and 'banking'
\moderncvcolor{green}                               % color options 'blue' (default), 'orange', 'green', 'red', 'purple', 'grey' and 'black'

% personal data
\name{Михаил}{Матросов}
\title{C++ разработчик}                               % optional, remove / comment the line if not wanted
\address{Россия, Москва} % optional, remove / comment the line if not wanted; the "postcode city" and "country" arguments can be omitted or provided empty
\phone[mobile]{+7~(926)~381~61~64}
\email{mikhail.matrosov@gmail.com}                               % optional, remove / comment the line if not wanted
\social[linkedin]{mikhail-matrosov}                        % optional, remove / comment the line if not wanted
\social[github]{mmatrosov}                              % optional, remove / comment the line if not wanted
\photo[64pt][0.4pt]{picture}                       % optional, remove / comment the line if not wanted; '64pt' is the height the picture must be resized to, 0.4pt is the thickness of the frame around it (put it to 0pt for no frame) and 'picture' is the name of the picture file
%\quote{First things first}                                 % optional, remove / comment the line if not wanted

\begin{document}
\maketitle

\section{Коротко о главном}

Большой опыт разработки приложений по обработке изображений на С++ в среде Visual Studio. Хорошее владение Matlab для прототипирования проектов. Много работал с библиотеками Intel IPP и OpenCV. Базовые представления об оптимизации многопоточных программ.

Большой опыт исследовательской работы в области обработки изображений, тональной компрессии и теории цвета. Тем не менее, больше заинтересован в развитии технических навыков, чем в продолжении научной работы.

Могу похвастаться аккаунтом на StackOverflow, расширением NativeViewer для Visual Studio.

\section{Навыки}

\subsection{Языки программирования}

C++, Matlab, C\#, JavaScript, HTML, win-batch, SQL

\subsection{Библиотеки и технологии}

boost, Intel IPP, Intel MKL, CGAL, OpenCV, ASP.NET, 

\subsection{Средства и технологии для разработки}

Visual Studio, SVN, GIT, LaTeX, 



\section{Научные и фундаментальные знания}

Обработка изображений~\cite{matrosov2009interactive,matrosov2009diploma}, теория цвета~\cite{matrosov2013correction}, компьютерное зрение, машинная графика.

Линейная алгебра.

\section{Опыт работы}

\cvitem{2009--2010}{
	Преподавал прак.
}

\section{Образование}
\cventry{2004--2009}{Специалист}{Московский Государственный Университет}{}{}{Факультет Вычислительной Математики и Кибернетики}
\cvitem{2004--2009}{
	\textbf{Московский Государственный Университет} \newline{} 
	Специалист, факультет Вычислительной Математики и Кибернетики
}

\section{Владение языками}

\cvitem{Английский}{Продвинутый. Свободное чтение и письмо на произвольные темы. Свободный диалог на технические темы.}
\cvitem{Русский}{Носитель.}

\section{Интересы}

\cvitem{}{Велосипед, паркур, сноуборд.}


\renewcommand{\bibliographyitemlabel}{[\arabic{enumiv}]}
\renewcommand{\refname}{Публикации}
\bibliographystyle{plain}
\bibliography{publications}


\end{document}
