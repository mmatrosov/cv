\documentclass[11pt,a4paper,final]{moderncv}

\usepackage[T2A]{fontenc}
% исходный текст в кодировки unicode
\usepackage[utf8]{inputenc}
% включаем поддержку русского языка
\usepackage[russian]{babel}

\renewcommand{\rmdefault}{cmr} % Шрифт с засечками
\renewcommand{\sfdefault}{cmss} % Шрифт без засечек
\renewcommand{\ttdefault}{cmtt} % Моноширинный шрифт

% moderncv themes
\moderncvstyle{classic}                             % style options are 'casual' (default), 'classic', 'oldstyle' and 'banking'
\moderncvcolor{green}                               % color options 'blue' (default), 'orange', 'green', 'red', 'purple', 'grey' and 'black'

% personal data
\name{\fontsize{1cm}{10em}\selectfont Михаил \vspace{0.5ex}}{Матросов}
\title{C++ разработчик}                               % optional, remove / comment the line if not wanted
\address{Россия, Москва} % optional, remove / comment the line if not wanted; the "postcode city" and "country" arguments can be omitted or provided empty
\phone[mobile]{+7~(926)~381-61-64}
\email{mikhail.matrosov@gmail.com}                               % optional, remove / comment the line if not wanted
\social[linkedin]{mmatrosov}                        % optional, remove / comment the line if not wanted
\social[github]{mmatrosov}                              % optional, remove / comment the line if not wanted
\extrainfo{\includegraphics[height=1em]{stackexchange.png} \httplink[Mikhail]{stackoverflow.com/users/261217/mikhail}}
\photo[64pt][0.4pt]{picture}                       % optional, remove / comment the line if not wanted; '64pt' is the height the picture must be resized to, 0.4pt is the thickness of the frame around it (put it to 0pt for no frame) and 'picture' is the name of the picture file
%\quote{First things first}                                 % optional, remove / comment the line if not wanted

\begin{document}
\maketitle

\section{Коротко о главном}

Большой опыт разработки приложений по обработке изображений на С++ в среде Visual Studio. Хорошее владение Matlab для прототипирования проектов. Много работал с библиотеками Intel IPP и OpenCV. Базовые представления об оптимизации многопоточных программ.

Большой опыт исследовательской работы в области обработки изображений, тональной компрессии и теории цвета. Тем не менее, больше заинтересован в развитии технических навыков, чем в продолжении научной работы.

Богатое прошлое в олимпиадном программировании. Поступил в МГУ без экзаменов благодаря диплому I степени на Всероссийской Олимпиаде по Информатике. Хорошее знание широкого класса алгоритмов и структур данных.

Могу похвастаться аккаунтом на \httplink[StackOverflow]{stackoverflow.com/users/261217/mikhail}, расширением \httplink[NativeViewer]{sourceforge.net/projects/nativeviewer/} для Visual Studio.

\section{Навыки}

\subsection{Языки программирования}

\cvitem{основные}{
	C++, Matlab
}
\cvitem{ещё знаю}{
	C\#, JavaScript, HTML, SQL
}

\subsection{Библиотеки и технологии}

\cvitem{основные}{
	OpenCV, Intel IPP, STL, Microsoft ConcRT
}
\cvitem{ещё умею}{
	boost, Intel MKL, MFC, CGAL, ASP.NET, jQuery
}

\subsection{Прикладные программы и системы}

\cvitem{основные}{
	Visual Studio, SVN, GIT, win-batch, Total Commander
}
\cvitem{ещё могу}{
	\LaTeX, Photoshop
}


\section{Научные и фундаментальные знания}

\cvitem{}{
	Обработка изображений~\cite{matrosov2009diploma}, теория цвета~\cite{matrosov2013correction}, компьютерное зрение, машинная графика.
}
\cvitem{}{
	Алгоритмы и структуры данных, вычислительная геометрия, линейная алгебра.
}


\section{Опыт работы}

\cvitem{c 2009}{
	\textbf{OctoNus Software Ltd.} \newline{} 
	Разработка алгоритмов и ПО. Основной опыт работы. Анализ задач клиентов, связанных с качеством изображений бриллиантов, разработка методик и алгоритмов для их решения. Реализация и оптимизация алгоритмов.
}
\cvitem{c 2009}{
	\textbf{Лаборатория компьютерной графики и мультимедиа ВМК МГУ} \newline{} 
	Исследовательская работа в качестве аспиранта. Тесно связана с работой в OctoNus.
}
\cvitem{2011}{
	\textbf{Кафедра АСВК факультета ВМК МГУ} \newline{} 
	Преподаватель практикума по С++ для студентов 3 курса.
}
\cvitem{2004--2006}{
	\textbf{Летняя Компьютерная Школа} \newline{} 
	Преподаватель группы С, вожатый. 
}

\section{Образование}

\cvitem{2009--2012}{
	\textbf{Московский Государственный Университет} \newline{} 
	Факультет Вычислительной Математики и Кибернетики \newline{} 
	Аспирант кафедры АСВК.
}

\cvitem{2004--2009}{
	\textbf{Московский Государственный Университет} \newline{} 
	Факультет Вычислительной Математики и Кибернетики \newline{} 
	Студент специалист.
}
\cvitem{2002--2003}{
	\textbf{Летняя Компьютерная Школа} \newline{} 
	Ученик групп С и А. Изучение широкого класса алгоритмов и структур данных.
}

\section{Владение языками}

\cvitem{Английский}{Продвинутый. Свободное чтение и письмо на произвольные темы. Свободный диалог на технические темы.}
\cvitem{Русский}{Носитель :)}


\section{Свои проекты}

\cvitem{с 2012}{
	\httplink[\textbf{NativeViewer}]{sourceforge.net/projects/nativeviewer/} \newline{} 
	Расширение Visual Studio для просмотра изображений OpenCV прямо во время отладки С++ кода. В отличие от Microsoft Image Watch работает для всех версий Visual Studio.
}


\section{Интересы}

\cvitem{}{Велосипед, паркур, сноуборд.}


\renewcommand{\bibliographyitemlabel}{[\arabic{enumiv}]}
\renewcommand{\refname}{Публикации}
\bibliographystyle{plain}
\bibliography{publications}


\end{document}
