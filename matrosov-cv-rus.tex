% Common header for both English and Russian versions

\documentclass[11pt,a4paper,final]{moderncv}

\usepackage[T2A]{fontenc}
\usepackage[utf8]{inputenc}
\usepackage[russian]{babel}
\usepackage{dashundergaps}

\renewcommand{\rmdefault}{cmr}  % Шрифт с засечками
\renewcommand{\sfdefault}{cmss} % Шрифт без засечек
\renewcommand{\ttdefault}{cmtt} % Моноширинный шрифт

% moderncv themes
\moderncvstyle{classic}  % style options are 'casual' (default), 'classic', 'oldstyle' and 'banking'
\moderncvcolor{green}  % color options 'blue' (default), 'orange', 'green', 'red', 'purple', 'grey' and 'black'

\newcommand{\StackExchange}{\includegraphics[height=1em]{stackexchange.png}}

% My custom httplink command with underlined text
\newcommand*{\myhttplink}[2][]{\href{http://#2}{\dotuline{#1}}}

\newcommand{\cvmonth}[1]{{\scriptsize/#1}}

% Language-agnostic personal data
\phone[mobile]{+7~(926)~381-61-64}
\email{mikhail.matrosov@gmail.com}
\social[linkedin]{mmatrosov}
\social[github]{mmatrosov}
\extrainfo{\StackExchange\httplink[Mikhail]{stackoverflow.com/users/261217/mikhail}}


% Language-specific personal data
\name{Михаил \vspace{0.5ex} \\}{Матросов}
\title{C++ разработчик}
\address{Россия, Москва}


\begin{document}
\maketitle
	
С++ джедай. Создаю ПО и умею работать в команде. Решаю вопросы и оптимизирую всё, что можно. Умею общаться с людьми и люблю учить людей. О себе, в трёх словах: C++, GIT, Visual Studio.

\section{Навыки}

\subsection{Языки программирования}

\cvitem{основные}{
	\textbf{C++}: с 2005, приложения по обработке изображений, многопоточные приложения, консольные утилиты, GUI на Qt и MFC, 3D-графика; проведение семинаров и докладов внутри компаний и на публичных конференциях. \newline
	\textbf{Matlab}: с 2007 по 2014, прототипирование алгоритмов и GUI, mex-интерфейсы.
}
\cvitem{вторичные}{
	\textbf{Python}, \textbf{C\#}, \textbf{JavaScript}, \textbf{HTML}, \textbf{SQL}: небольшие утилиты, домашние проекты, простые web-приложения.
}

\subsection{Библиотеки и технологии}

\cvitem{основные}{
	\textbf{STL}, \textbf{boost}, \textbf{Qt}, \textbf{OpenCV}, \textbf{Intel IPP}: множество проектов по обработке изображений, проекты с использованием комбинаторики, численных методов, теории графов, 2D и 3D геометрии.
}
\cvitem{вторичные}{
	\textbf{Microsoft ConcRT}, \textbf{Intel MKL}, \textbf{CGAL}, \textbf{ASP.NET}, \textbf{jQuery}: поверхностные знания, использовались только в нескольких проектах.
}

\subsection{Прикладные программы и системы}

\cvitem{основные}{
	\textbf{Windows}, \textbf{Visual Studio}, \textbf{SmartGit}, \textbf{GIT}, \textbf{SVN}, \textbf{win-batch}, \textbf{NuGet}
}
\cvitem{вторичные}{
	\textbf{Unix}, \textbf{\LaTeX}, \textbf{Photoshop}, \textbf{bash}
}

\subsection{Научные и фундаментальные знания}

\cvitem{}{
	Обработка изображений~\cite{matrosov2009diploma}, теория цвета~\cite{matrosov2013correction}, компьютерное зрение, машинная графика, алгоритмы, структуры данных.
}


\section{Опыт работы}

\cvitem{с~2014\cvmonth{12}}{
	\textbf{Align Technology Inc}, \emph{Старший разработчик}. \newline 
	Развитие 3D-приложения для создания персонализированных ортодонтических протезов. Курирование разработки новых продуктов для выхода на смежные рынки. Прокачивание процесса разработки: code review, GIT workflow, coding standards. 
}
\cvitem{с~2013\cvmonth{10} по~2014\cvmonth{10}}{
	\textbf{OctoNus Software Ltd, проект Digital Microscope}, \newline
	\emph{Разработчик}. \newline
	Сделал прототип системы виртуальной навигации по набору фотографий объекта, в системе шесть степеней свободы. Разработал алгоритм цветокоррекции через непрерывную трансформацию цветовых пространств, построенную на основе ряда опорных точек. Занимался разработкой и поддержкой Qt-приложения для отображения и обработки видео-потока с сетевых камер. Развивал систему плагинов и SDK. Улучшил структуру взаимодействия модулей приложения. 
}
\cvitem{с~2008\cvmonth{08} по~2013\cvmonth{09}}{
	\textbf{OctoNus Software Ltd}, \emph{Разработчик}. \newline 
	Анализ проблем с ПО для получения фотографий ювелирной продукции. Разработка и внедрение алгоритмов улучшения изображений. Выполнил эффективную реализацию на С++ с использованием Intel IPP и Microsoft ConcRT ряда алгоритмов тональной компрессии, расширения глубины резкости, цветокоррекции, 3D-реконструкции и объединения 3D-моделей. Для всех описанных алгоритмов предварительно собрал необходимые данные у партнёров, выполнил анализ существующих подходов, прототипировал решения на Matlab. Реализованные алгоритмы работают в реальном времени и используются для оценки качества ювелирной продукции.
}
\cvitem{с~2009\cvmonth{10} по~2013\cvmonth{10}}{
	\textbf{Лаборатория Компьютерной Графики и Мультимедиа ВМК МГУ}, \emph{Исследователь} \newline 
	Формально значился аспирантом, фактически вёл исследовательскую деятельность по проектам в OctoNus (см. выше).
}
\cvitem{с~2011\cvmonth{02} по~2011\cvmonth{05}}{
	\textbf{Кафедра АСВК факультета ВМК МГУ}, \newline
	\emph{Преподаватель} \newline 
	Практикум по С++ для студентов 3-го курса.
}
\cvitem{с~2004 по~2006}{
	\textbf{Летняя Компьютерная Школа}, \newline 
	\emph{Преподаватель, Вожатый} \newline 
	Теоретические лекции и практикум для группы С.	
}


\section{Свои проекты и вклады}

\cvitem{2015\cvmonth{03}}{
	\myhttplink[\textbf{boost.python}]{boost.org/libs/python}, \emph{Участник} \newline 
	\myhttplink[Pull request \#15]{github.com/boostorg/python/pull/15}. Fix \#11100 and \#8058: binary compatibility and leaked file handle in exec\_file().
}
\cvitem{2015\cvmonth{02}}{
	\myhttplink[\textbf{Конференция C++ Russia}]{meetingcpp.ru/?page_id=233}, \emph{Докладчик} \newline 
	Доклад \myhttplink[<<С++ without new and delete>>]{meetingcpp.ru/?page_id=608}.
}
\cvitem{2014\cvmonth{10}}{
	\myhttplink[\textbf{Встреча C++ User Group, Russia}]{meetingcpp.ru/?p=250}, \emph{Докладчик} \newline 
	Доклад \myhttplink[<<С++ без new и delete>>]{www.slideshare.net/sermp/c-without-new-and-delete-russian-c-user-group}.
}
\cvitem{с~2012\cvmonth{06} по~2013\cvmonth{07}}{
	\myhttplink[\textbf{NativeViewer}]{sourceforge.net/projects/nativeviewer/}, \emph{Автор} \newline 
	Расширение Visual Studio для просмотра изображений OpenCV прямо во время отладки С++ кода. В отличие от Microsoft Image Watch, работает для версий Visual Studio начиная с 2003.
}
\cvitem{2012\cvmonth{02}}{
	\myhttplink[\textbf{OpenCV}]{code.opencv.org/users/915}, \emph{Участник} \newline 
	\myhttplink[Патч \#1641]{code.opencv.org/issues/1641}. Discrete Voronoi diagram: returning closest pixel instead of connected component in distanceTransform.
}
\cvitem{с~2010\cvmonth{01}}{
	\myhttplink[\textbf{StackOverflow}]{stackoverflow.com/users/261217/mikhail}, \emph{Участник} \newline 
	Более 7500 репутации, более 50 принятых ответов. \newline
	Топ 10\% по тэгам \texttt{c++}, \texttt{algorithm} и \texttt{image-processing}. \newline
	Топ 20\% по тэгам \texttt{matlab} и \texttt{opencv}.
}

\section{Тренинги и сертификаты}

\cvitem{2014\cvmonth{06}}{
	\myhttplink[\textbf{An Overview of the New C++ (C++11/14)}]{www.aristeia.com/C++11.html} \newline 
	Технический тренинг от Скотта Мейерса.
}

\section{Образование}

\cvitem{с~2009 по~2012}{
	\textbf{Московский Государственный Университет} \newline 
	Факультет Вычислительной Математики и Кибернетики \newline
	\emph{Аспирант кафедры АСВК}.
}
\cvitem{с~2004 по~2009}{
	\textbf{Московский Государственный Университет} \newline 
	Факультет Вычислительной Математики и Кибернетики \newline
	\emph{Студент, специалист}. \newline
	Диплом~\cite{matrosov2009diploma}. Поступил без экзаменов благодаря диплому I степени на \myhttplink[XVI Всероссийской Олимпиаде по Информатике]{neerc.ifmo.ru/school/archive/2003-2004/ru-olymp-roi-2004-standings.html}.
}
\cvitem{с~2001 по~2003}{
	\textbf{Летняя Компьютерная Школа} \newline
	\emph{Ученик групп С и А}. \newline
	Изучение широкого класса алгоритмов и структур данных.
}

\section{Владение языками}

\cvitem{Английский}{Продвинутый. Свободное чтение и письмо на произвольные темы. Свободный диалог на технические темы.}
\cvitem{Русский}{Носитель. Грамотное письмо.}



\renewcommand{\bibliographyitemlabel}{[\arabic{enumiv}]}
\renewcommand{\refname}{Публикации}
\bibliographystyle{plain}
\bibliography{publications}


\end{document}
