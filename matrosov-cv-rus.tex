% Common header for both English and Russian versions

\documentclass[11pt,a4paper,final]{moderncv}

\usepackage[T2A]{fontenc}
\usepackage[utf8]{inputenc}
\usepackage[russian]{babel}
\usepackage{dashundergaps}

\renewcommand{\rmdefault}{cmr}  % Шрифт с засечками
\renewcommand{\sfdefault}{cmss} % Шрифт без засечек
\renewcommand{\ttdefault}{cmtt} % Моноширинный шрифт

% moderncv themes
\moderncvstyle{classic}  % style options are 'casual' (default), 'classic', 'oldstyle' and 'banking'
\moderncvcolor{green}  % color options 'blue' (default), 'orange', 'green', 'red', 'purple', 'grey' and 'black'

\newcommand{\StackExchange}{\includegraphics[height=1em]{stackexchange.png}}

% My custom httplink command with underlined text
\newcommand*{\myhttplink}[2][]{\href{http://#2}{\dotuline{#1}}}

\newcommand{\cvmonth}[1]{{\scriptsize/#1}}

% Language-agnostic personal data
\phone[mobile]{+7~(926)~381-61-64}
\email{mikhail.matrosov@gmail.com}
\social[linkedin]{mmatrosov}
\social[github]{mmatrosov}
\extrainfo{\StackExchange\httplink[Mikhail]{stackoverflow.com/users/261217/mikhail}}


% Language-specific personal data
\name{Михаил \vspace{0.5ex} \\}{Матросов}
\title{C++ разработчик}
\address{Россия, Москва}


\begin{document}
\maketitle
	
С++ джедай. Создаю ПО и умею работать в команде. Решаю вопросы и оптимизирую всё, что можно. Умею общаться с людьми и люблю учить людей. О себе, в трёх словах: C++, Git, Visual Studio.

\section{Навыки}

\subsection{Языки программирования}

\cvitem{основные}{
	\textbf{C++}: с 2005, приложения по обработке изображений, многопоточные приложения, консольные утилиты, GUI на Qt и MFC, 3D-графика, CAD системы; проведение семинаров и докладов внутри компаний и на публичных конференциях. \newline
	\textbf{Matlab}: с 2007 по 2014, прототипирование алгоритмов и GUI, mex-интерфейсы.
}
\cvitem{вторичные}{
	\textbf{Python}, \textbf{C\#}, \textbf{JavaScript}, \textbf{SQL}: небольшие утилиты, домашние проекты, простые web-приложения.
}

\subsection{Библиотеки и технологии}

\cvitem{основные}{
	\textbf{STL}, \textbf{boost}, \textbf{OpenCV}, \textbf{Qt}, \textbf{Intel IPP}: множество проектов по обработке изображений, проекты с использованием комбинаторики, численных методов, теории графов, 2D и 3D геометрии.
}
\cvitem{вторичные}{
	\textbf{Microsoft ConcRT}, \textbf{Intel MKL}, \textbf{OpenSceneGraph}, \textbf{CGAL}, \textbf{ASP.NET}, \textbf{jQuery}: поверхностные знания, использовались только в нескольких проектах.
}

\subsection{Прикладные программы и системы}

\cvitem{основные}{
	\textbf{Windows}, \textbf{Visual Studio}, \textbf{SmartGit}, \textbf{Git}, \textbf{SVN}, \textbf{Total Commander}, \textbf{Conan}
}
\cvitem{вторичные}{
	\textbf{Unix}, \textbf{\LaTeX}, \textbf{Photoshop}, \textbf{bash}
}

\subsection{Научные и фундаментальные знания}

\cvitem{}{
	Обработка изображений~\cite{matrosov2009diploma}, теория цвета~\cite{matrosov2013correction}, компьютерное зрение, машинная графика, алгоритмы, структуры данных.
}

\newpage  % Чтобы название секции не болталось одиноко в конце страницы
\subsection{Продуктивность}

\cvitem{}{
	Слепая печать на английской и русской раскладках. Чтобы не перемещать лишний раз правую руку на стрелочную часть клавиатуры для навигации во время набора текста, сделал утилиту KeysRemapper (см. секцию <<\hyperref[sec:contributions]{Проекты}>>).
}


\section{Опыт работы}

\cvitem{с~2019 по~2020}{
	\textbf{Высшая Школа Экономики}, \emph{Преподаватель}. \newline
	Проведение семинарских занятий по С++ в рамках курса <<Основы и методология программирования>> бакалаврской программы <<Прикладная математика и информатика>> на Факультете Компьютерных Наук.\newline
	Объяснение базовых концепций языка (в дополнение к лекциям) и практик программирования. Разбор домашних заданий, ответы на вопросы.
}
\cvitem{с~2020\cvmonth{02} по~2020\cvmonth{09}}{
	\textbf{Align Technology Inc}, \emph{Expert developer, TPE team}. \newline 
	Внедрение практики юнит тестирования в процесс разработки. Создание инструкций и гайдов, проведение лекций, переговоры с командами, написание вспомогательных утилит, регулярное участие в ревью кода.\newline 
	Создание целостной формальной документации для обширной и плохо понятной части приложения, исправление поведения приложения для достижения консистентности. Использование принципа "documentation as a code": документация хранится в формате AsciiDoc и автоматически конвертируется в Word с помощью Pandoc.
}
\cvitem{с~2019\cvmonth{01} по~2020\cvmonth{01}}{
	\textbf{Align Technology Inc}, \emph{Expert developer, BMS team}. \newline 
	Выделение части монолита, поддерживаемой нашей командой, в отдельный веб сервис. Сервис хостится в AWS под Linux. Монолит запускается на десктопах под Windows. Реализация облачного логгирования в Splunk.
}
\cvitem{с~2018\cvmonth{02} по~2018\cvmonth{12}}{
	\textbf{Align Technology Inc}, \emph{Expert developer, 3D platform team}. 	\newline 
	Продолжение работы в команде 3D платформы. Прокачивание процесса работы со сторонними библиотеками на C++: замена существующего самописного решения на пакетный менеджер Conan. Выступаю техническим лидером и ментором для команды джуниор разработчиков в рамках задачи поддержки кросс-платформенности для ряда приложений компании. Апгрейд компилятора с Visual Studio 2013 до Visual Studio 2017.
}
\cvitem{с~2017\cvmonth{01} по~2018\cvmonth{02}}{
	\textbf{Align Technology Inc}, \emph{Technical manager, 3D platform team}. \newline 
	Полное переключение на глобальные задачи ПО, касающиеся всех разработчиков. Миграция на 64-битную платформу. Поддержка локализации. Поддержка Unicode. Унификация форматирования кода. Архитектурная переработка: разделение модели и представления. Видение дальнейшего развития ПО. Плюс прокачивание процесса разработки как и раньше.
}
\cvitem{с~2016\cvmonth{02} по~2016\cvmonth{12}}{
	\textbf{Align Technology Inc}, \emph{Technical manager, BMS team}. \newline 
	Бизнес анализ: сбор требований по новой функциональности со всех позиций, включая пользователей ПО и докторов. Межкомандное взаимодействие для успешной интеграции функционала во все компоненты системы. Продвижение компании: публичные доклады и публикации. Плюс всё то же, что и раньше.	
}
\cvitem{с~2014\cvmonth{12} по~2016\cvmonth{02}}{
	\textbf{Align Technology Inc}, \emph{Старший разработчик, BMS team}. \newline 
	Развитие CAD-приложения для планирования ортодонтического лечения. Курирование разработки функционала для выхода на подростковый рынок. Прокачивание процесса разработки: code review, Git workflow, стандарты кодирования, статический анализ кода. Геометрические алгоритмы с твёрдыми телами в 3D. Модернизация кода.
}
\cvitem{с~2013\cvmonth{10} по~2014\cvmonth{10}}{
	\textbf{OctoNus Software Ltd, проект Digital Microscope}, \newline
	\emph{Разработчик}. \newline
	Сделал прототип системы виртуальной навигации по набору фотографий объекта, в системе шесть степеней свободы. Разработал алгоритм цветокоррекции через непрерывную трансформацию цветовых пространств, построенную на основе ряда опорных точек. Занимался разработкой и поддержкой Qt-приложения для отображения и обработки видео-потока с сетевых камер. Развивал систему плагинов и SDK. Улучшил структуру взаимодействия модулей приложения. 
}
\cvitem{с~2008\cvmonth{08} по~2013\cvmonth{09}}{
	\textbf{OctoNus Software Ltd}, \emph{Разработчик}. \newline 
	Анализ проблем с ПО для получения фотографий ювелирной продукции. Разработка и внедрение алгоритмов улучшения изображений. Выполнил эффективную реализацию на С++ с использованием Intel IPP и Microsoft ConcRT ряда алгоритмов тональной компрессии, расширения глубины резкости, цветокоррекции, 3D-реконструкции и объединения 3D-моделей. Разработал алгоритм устранения дыхания объектива камеры. Для всех описанных алгоритмов предварительно собрал необходимые данные у партнёров, выполнил анализ существующих подходов, прототипировал решения на Matlab. Реализованные алгоритмы работают в реальном времени и используются для оценки качества ювелирной продукции.
}
\cvitem{с~2009\cvmonth{10} по~2013\cvmonth{10}}{
	\textbf{Лаборатория Компьютерной Графики и Мультимедиа ВМК МГУ}, \emph{Исследователь} \newline 
	Формально значился аспирантом, фактически вёл исследовательскую деятельность по проектам в OctoNus (см. выше).
}
\cvitem{с~2011\cvmonth{02} по~2011\cvmonth{05}}{
	\textbf{Кафедра АСВК факультета ВМК МГУ}, \newline
	\emph{Преподаватель} \newline 
	Практикум по С++ для студентов 3-го курса.
}
\cvitem{с~2004 по~2006}{
	\textbf{Летняя Компьютерная Школа}, \newline 
	\emph{Преподаватель, Вожатый} \newline 
	Теоретические лекции и практикум для группы С.	
}


\section{Доклады}

Доклады, отмеченные символом $\star$, заслуживают быть выделенными.

\cvitem{$\star$ 2020\cvmonth{07}}{
	\myhttplink[\textbf{C++ Russia 2020}]{https://cppconf-moscow.ru/} \newline 
	\myhttplink[<<Как объявить константу в С++?>>]{https://cppconf-moscow.ru/2020/msk/talks/50qcqrqjnqti2i3tetrljs/}
}
\cvitem{$\star$ 2019\cvmonth{11}}{
	\myhttplink[\textbf{C++ Russia 2019 Piter}]{https://cppconf-piter.ru/} \newline 
	\myhttplink[<<Спецификаторы, квалификаторы и шаблоны>>]{https://cppconf-piter.ru/2019/spb/talks/6jki3jwkruhn4sjasbshmb/}
}
\cvitem{$\star$ 2019\cvmonth{04}}{
	\myhttplink[\textbf{C++ Russia 2019 Moscow}]{https://2019.cppconf-moscow.ru/} \newline 
	\myhttplink[<<Как мы апгрейдили компилятор и поддерживали кроссплатформенность>>]{https://2019.cppconf-moscow.ru/talks/1hkys3ywmkmnltmfekuvfr/}
}
\cvitem{$\star$ 2018\cvmonth{10}}{
	\myhttplink[\textbf{SECR 2018}]{http://2018.secr.ru/lang/ru/} \newline 
	\myhttplink[<<Как не потонуть в пучине легаси>>]{https://2018.secrus.org/program/submitted-presentations/how-not-to-sink-in-legacy/}
}
\cvitem{2018\cvmonth{02}}{
	\myhttplink[\textbf{C++ Russia 2018}]{http://2018.cppconf.ru} \newline 
	\myhttplink[<<Versatile C++ applied>>]{https://2018.cppconf-piter.ru/talks/mikhail-matrosov.html}
}
\cvitem{$\star$ 2017\cvmonth{10}}{
	\myhttplink[\textbf{SECR 2017}]{http://2017.secr.ru/lang/ru/} \newline 
	\myhttplink[<<Reverting a merge>>]{http://2017.secr.ru/lang/en/program/submitted-presentations/reverting-a-merge-without-console}
}
\cvitem{$\star$ 2017\cvmonth{09}}{
	\myhttplink[\textbf{CppCon 2017}]{https://cppcon.org/cppcon-2017-program/} \newline 
	\myhttplink[<<Refactor or die>>]{https://youtu.be/fzmjXK9JZ9o}
}
\cvitem{2017\cvmonth{04}}{
	\myhttplink[\textbf{SECON 2017}]{http://2017.secon.ru} \newline 
	\myhttplink[<<Повседневный С++: алгоритмы и итераторы>>]{http://2017.secon.ru/reports/povsednevnyy-s-algoritmy-i-iteratory}
}
\cvitem{$\star$ 2017\cvmonth{02}}{
	\myhttplink[\textbf{C++ Russia 2017}]{http://2017.cppconf.ru} \newline 
	\myhttplink[<<Повседневный С++: алгоритмы и итераторы>>]{http://2017.cppconf.ru/talks/mikhail-matrosov}
}
\cvitem{2017\cvmonth{02}}{
	\myhttplink[\textbf{C++ CoreHard 2017}]{http://corehard.by/category/corehard-conf-winter-2017/} \newline 
	\myhttplink[<<Повседневный С++: алгоритмы и итераторы>>]{http://corehard.by/2017/02/20/day-to-day-c-algorithms-and-iterators/}
}
\cvitem{$\star$ 2016\cvmonth{02}}{
	\myhttplink[\textbf{C++ Russia 2016}]{http://cpp-russia.ru/?page_id=936} \newline 
	\myhttplink[<<Повседневный С++: boost и STL>>]{http://cpp-russia.ru/?page_id=999}
}
\cvitem{2016\cvmonth{02}}{
	\myhttplink[\textbf{C++ CoreHard 2016}]{http://corehard.by/category/corehard-conf-2016/} \newline 
	\myhttplink[<<Повседневный С++: boost и STL>>]{http://corehard.by/2016/02/15/conf2016-daily-cpp/}
}
\cvitem{2015\cvmonth{10}}{
	\myhttplink[\textbf{CEE-SECR 2015}]{http://2015.secr.ru/} \newline 
	\myhttplink[<<Повседневный С++>>]{http://2015.secr.ru/lang/ru/program/submitted-presentations/daily-cpp}
}
\cvitem{$\star$ 2015\cvmonth{02}}{
	\myhttplink[\textbf{C++ Russia 2015}]{http://cpp-russia.ru/?page_id=233} \newline 
	\myhttplink[<<С++ without new and delete>>]{http://cpp-russia.ru/?page_id=608}
}
\cvitem{2014\cvmonth{10}}{
	\myhttplink[\textbf{Встреча C++ User Group, Russia}]{http://cpp-russia.ru/?p=286} \newline 
	\myhttplink[<<С++ без new и delete>>]{http://www.slideshare.net/sermp/c-without-new-and-delete-russian-c-user-group}
}


\section{Свои проекты и вклады}\label{sec:contributions}

\cvitem{с~2019\cvmonth{02} по~2019\cvmonth{5}}{
	\myhttplink[\textbf{Основы разработки на С++: чёрный пояс}]{https://www.coursera.org/learn/c-plus-plus-black}, \newline \emph{Соавтор} \newline 
	Курс по С++ на курсере, созданный совсемстно с Яндексом. Пятый из пяти в рамках специализации \myhttplink[Искусство разработки на современном C++]{https://www.coursera.org/specializations/c-plus-plus-modern-development}. В своём блоке рассказываю про undefined behavior и разбираю большую практическую задачу по созданию движка для электронной таблицы (типа Excel, или бэкенда для Google Spreadsheets).
}
\cvitem{с~2018\cvmonth{09} по~2018\cvmonth{12}}{
	\myhttplink[\textbf{Основы разработки на С++: коричневый пояс}]{https://www.coursera.org/learn/c-plus-plus-brown}, \newline \emph{Соавтор} \newline 
	Курс по С++ на курсере, созданный совсемстно с Яндексом. Четвёртый из пяти в рамках специализации \myhttplink[Искусство разработки на современном C++]{https://www.coursera.org/specializations/c-plus-plus-modern-development}. Имеет рейтинг 5.0 через три месяца после запуска. В своём блоке рассказываю про устройство и использование умных указателей.
}
\cvitem{2017\cvmonth{05}}{
	\myhttplink[\textbf{DllDispatcher}]{https://github.com/mmatrosov/DllDispatcher}, \emph{Автор} \newline 
	Инструмент, позволяющий ассоциировать dll файлы с различными приложениями в зависимости от битности dll. Одно приложение для 32-битных dll и другое для 64-битных dll. Сделано для того, чтобы проассоциировать dll с соответствующими версиями Dependency Walker.
}
\cvitem{2015\cvmonth{03}}{
	\myhttplink[\textbf{boost.python}]{http://boost.org/libs/python}, \emph{Участник} \newline 
	\myhttplink[Pull request \#15]{http://github.com/boostorg/python/pull/15}. Fix \#11100 and \#8058: binary compatibility and leaked file handle in exec\_file().
}
\cvitem{2014\cvmonth{10}}{
	\myhttplink[\textbf{KeysRemapper}]{http://github.com/mmatrosov/KeysRemapper}, \emph{Автор} \newline 
	Утилита по нажатию клавиши CapsLock превращает часть буквенной клавиатуры в кнопки навигации: стрелки, Ins, Del, Home, End, и т.д. Эдакий упрощённый Vim, но работает во всей ОС, а не только в текстовых редакторах.
}
\cvitem{с~2012\cvmonth{06} по~2013\cvmonth{07}}{
	\myhttplink[\textbf{NativeViewer}]{http://sourceforge.net/projects/nativeviewer/}, \emph{Автор} \newline 
	Расширение Visual Studio для просмотра изображений OpenCV прямо во время отладки С++ кода. В отличие от Microsoft Image Watch, работает для версий Visual Studio начиная с 2003.
}
\cvitem{2012\cvmonth{02}}{
	\myhttplink[\textbf{OpenCV}]{http://code.opencv.org/users/915}, \emph{Участник} \newline 
	\myhttplink[Патч \#1641]{http://code.opencv.org/issues/1641}. Discrete Voronoi diagram: returning closest pixel instead of connected component in distanceTransform.
}
\cvitem{с~2010\cvmonth{01}}{
	\myhttplink[\textbf{StackOverflow}]{http://stackoverflow.com/users/261217/mikhail}, \emph{Участник} \newline 
	Более 15k репутации, \myhttplink[более 90 принятых ответов]{http://stackoverflow.com/search?q=user:261217+isaccepted:yes}. \newline
	Топ 10\% по тэгам \texttt{c++}, \texttt{algorithm} и \texttt{image-processing}. \newline
	Топ 20\% по тэгам \texttt{matlab} и \texttt{opencv}.
}

\section{Тренинги и сертификаты}

\cvitem{2020\cvmonth{02}}{
	\myhttplink[\textbf{SOLID Principles of Object-Oriented Design and Architecture}]{https://www.udemy.com/course/solid-principles-object-oriented-design-architecture/} \newline 
	 \myhttplink[Ссылка на сертификат]{http://ude.my/UC-0d85d789-0b8b-4e9b-956f-271559b15b1c}.
}
\cvitem{2019\cvmonth{04}}{
	\myhttplink[\textbf{The Bits and Bytes of Computer Networking}]{https://www.coursera.org/learn/computer-networking} \newline 
	Базовый курс по сетям от гугла. \myhttplink[Ссылка на сертификат]{https://www.coursera.org/account/accomplishments/verify/D6G7BH6YMS8N}.
}
\cvitem{2014\cvmonth{06}}{
	\myhttplink[\textbf{An Overview of the New C++ (C++11/14)}]{http://www.aristeia.com/C++11.html} \newline 
	Технический тренинг от Скотта Мейерса.
}

\section{Образование}

\cvitem{с~2009 по~2012}{
	\textbf{Московский Государственный Университет} \newline 
	Факультет Вычислительной Математики и Кибернетики \newline
	\emph{Аспирант кафедры АСВК}.
}
\cvitem{с~2004 по~2009}{
	\textbf{Московский Государственный Университет} \newline 
	Факультет Вычислительной Математики и Кибернетики \newline
	\emph{Студент, специалист}. \newline
	Диплом~\cite{matrosov2009diploma}. Поступил без экзаменов благодаря диплому I степени на \myhttplink[XVI Всероссийской Олимпиаде по Информатике]{http://neerc.ifmo.ru/school/archive/2003-2004/ru-olymp-roi-2004-standings.html}.
}
\cvitem{с~2001 по~2003}{
	\textbf{Летняя Компьютерная Школа} \newline
	\emph{Ученик групп С и А}. \newline
	Изучение широкого класса алгоритмов и структур данных.
}

\section{Владение языками}

\cvitem{Английский}{Продвинутый.}
\cvitem{Русский}{Носитель.}



\renewcommand{\bibliographyitemlabel}{[\arabic{enumiv}]}
\renewcommand{\refname}{Публикации}
\bibliographystyle{plain}
\bibliography{publications}


\end{document}
